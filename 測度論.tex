\documentclass[12pt]{jsarticle}
\usepackage{amssymb}
\usepackage{enumerate}
\usepackage{documentation}
\usepackage[dvipdfmx]{graphics}
\usepackage{amsmath}
\usepackage{mathtools}
\usepackage{amsthm}
\usepackage{tikz}
\usetikzlibrary{intersections,calc,arrows.meta}
\theoremstyle{definition}
\newtheorem{ex}{例}
\newtheorem{theorem}{定理}
\newtheorem*{theorem*}{定理}
\newtheorem{definition}[theorem]{定義}
\newtheorem*{definition*}{定義}
\renewcommand{\proofname}{\textbf{証明}}
\begin{document}
    \section*{問題1}
    $X$を全体集合、$\mathcal{A}$を$X$の部分集合族、$A$をある集合とし、$A,X \in \mathcal{A}$ とする。
    また、ある集合$B,C$に対して$A=B\oplus C$であるとし、任意の$\mathcal{A}$の元$D$に対して,$A \cap D=A$または
    $A \cap D=\varnothing$であるとする。(Aは分割できない的な状況にするための条件)
    このとき、$B \notin \sigma ( \mathcal{A})$かつ$C \notin \sigma ( \mathcal{A})$であることを示せ。
    ($\sigma ( \mathcal{A})$:$ \mathcal{A}$を含む最小の$\sigma$代数)
    \section*{解答}
    $\sigma ( \mathcal{A})$は$\mathcal{A}$を含むすべての$\sigma$代数の共通部分をとったものと一致することを用いると、
    $B,C$を含まないような$\mathcal{A}$を含む$\sigma$代数の存在を示せば十分であることがわかる。
    $\mathcal{B}=\lbrace\cup_{n=1}^\infty A_n\lvert A_n\in \mathcal{A}$または${A_n}^c\in \mathcal{A}\rbrace$とすると、これは$\mathcal{A}$を含む$\sigma$代数
    であることがわかるが、これは明らかに$B,C$を含まない。
\end{document}